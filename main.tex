\documentclass[12pt]{article}
\usepackage[utf8]{inputenc}
\usepackage[dvipsnames]{xcolor}
\usepackage[leqno]{amsmath}
\usepackage{amssymb}
%\usepackage{units}
\usepackage{wallpaper}
\usepackage{newtons-notebook}
\usepackage{booktabs}
\usepackage{float}
\usepackage{xurl}
\usepackage{listings}
%\usepackage{tocloft}
\usepackage{amsmath}
\usepackage{amsthm}
\usepackage{amsfonts}
\usepackage{amssymb}

\usepackage{tabu}
%\usepackage{multirow}
\usepackage{array}

\usepackage[english]{babel}
\usepackage{longtable}
%\usepackage[table]{xcolor} 
%\usepackage{indentfirst}

%\usepackage{gensymb}

\usepackage{hyperref}
\usepackage{color}
\usepackage[table,xcdraw]{xcolor}

\usepackage[font=small,skip=0pt]{caption}

\usepackage{graphicx}
\graphicspath{ {./assets/} }

\usepackage{setspace}
\usepackage{titlesec}

\usepackage{wrapfig}

% https://tex.stackexchange.com/questions/19660/how-to-make-the-size-of-pdf-output-wider
\setlength{\paperwidth}{8.75in} % set dimension of \paperwidth to add .125 inch on either sidev
\addtolength{\paperheight}{0.25in} % enlarge \paperheight by .25 inch

\titlespacing\section{0pt}{12pt plus 4pt minus 2pt}{0pt}
\titlespacing\subsection{0pt}{12pt plus 4pt minus 2pt}{0pt}
\titlespacing\subsubsection{0pt}{12pt plus 4pt minus 2pt}{0pt}

%\renewcommand{\cftpartfont}{\Large\bfseries}
%\renewcommand{\cftsecfont}{\normalsize}

% deals with space between paragraphs
\setlength{\parskip}{\baselineskip}
\renewcommand{\baselinestretch}{1.4}

% Margin Formatting https://en.wikibooks.org/wiki/LaTeX/Page_Layout
\addtolength{\hoffset}{0in}
\addtolength{\voffset}{0in}
\addtolength{\oddsidemargin}{-.125in}
\addtolength{\textwidth}{0.5in}
\addtolength{\topmargin}{-.125in}

% Theorem Enumeration https://tex.stackexchange.com/questions/371731/versioning-theorem-numbers-with-amsthm
\newcounter{dfnmain}[section]
\newtheorem{dfninner}{Definition}[dfnmain]
\makeatletter
\renewcommand{\thedfninner}{%
  \arabic{dfnmain}.\@arabic{\numexpr\value{dfninner}-1\relax}%
}
\makeatother
\newenvironment{dfn}
 {\stepcounter{dfnmain}\dfninner}
 {\enddfninner}
\newenvironment{dfn*}
 {\dfninner}
 {\enddfninner}
 
% Example formatting https://tex.stackexchange.com/questions/357810/math-example-formatting
\newenvironment{exm}[1]{%
  \par  % start a new paragraph
  \bigskip  % insert some vertical whitespace
  \noindent % no paragraph indentation
  \textbf{Example #1}}{%
  \par%\bigskip % insert another paragraph break and more vert. whitespace
}

\usepackage{tikz}

\newtheorem{theorem}{Theorem}

\theoremstyle{definition}
\newtheorem{definition}{Definition}

\newtheorem{problem}{Problem}

\newtheorem*{solution}{Solution}

\newtheorem*{remark}{Remark}

\newtheorem{assumption}{Assumption}

\begin{document}
\setcounter{tocdepth}{1}

% v COMMENT THIS OUT BEFORE SENDING THE FINAL VERSION TO PRINT, THEY DON'T WANT THE COVERS
% \nnimagepage{2020_Front_Cover.pdf}

\nnwallpaper{2020_Intro_Section_Page_Border.pdf}

\begin{figure}[H]
    \centering \includegraphics[scale=.9]{newton.jpg}
\end{figure}

\begin{center}
    \textit{To explain all nature is too difficult a task\\
    for any one man or even for any one age.\\
    'Tis much better to do a little with certainty\\
    and leave the rest for others that come after you.}\\
    	$\sim$Isaac Newton
\end{center}

\newpage
\begin{spacing}{1}
\tableofcontents
\end{spacing}

\begin{figure}[H]
    \centering
    \vspace*{75pt}
    \includegraphics[scale=1.25]{newtons_notebook_fibonacci_spiral.png}
\end{figure}

\newpage
\subsection*{Mission Statement}
\textit{Newton’s Notebook: The Haverford School STEM Journal} is designed to enhance the interests, talents, and achievements of individuals in mathematics and science and to promote the work of those most passionate about these disciplines. The following articles were written by members and friends of the Haverford community and edited by the \textit{Notebook staff}. We hope these articles inspire readers to further discover the universally beautiful realm of STEM exploration. 
\subsection*{Staff}
{\centering{}
    \textbf{Editors-in-Chief:} Gary Gao '21, Mitav Nayak '22
    \\
    \textbf{Designer-in-Chief:} Mitav Nayak '22
    \\ 
    \textbf{Assistant Editors:} Brian Williams '21, Adamya Aggarwal '22
    \\
    \textbf{Faculty Advisor:} Dr.\ Mark Gottlieb
    \\}
\subsection*{Featured Polymath: Richard Phillips Feynman (1918-1988)}
Richard Phillips Feynman was an American theoretical physicist known for his work on quantum mechanics, quantum electrodynamics, and particle physics. Born in Queens, New York, in 1918, Feynman excelled in math and science in high school. He attended the Massachusetts Institute of Technology and graduated with a bachelor’s degree in physics in 1939. In early 1943, Feynman was recruited by Robert Oppenheimer to work as a researcher on the Manhattan Project developing the atomic bomb. Soon after his work, Feynman took up a position as a physics professor at Cornell University. There he developed Feynman diagrams, a system of visually representing mathematical expressions describing the interaction of subatomic particles. In 1951, Feynman took up a position at the California Institute of Technology after taking a year-long sabbatical in Brazil. At Caltech, Feynman explored superfluidity and superconductivity, as well as conceiving the possibility of quantum computers. In the early 1960’s, Feynman produced \textit{The Feynman Lectures on Physics}, a lecture series outlining undergraduate physics concepts still used by university students to this day. In 1965, Feynman and two colleagues were awarded the Nobel Prize in Physics for their work on quantum electrodynamics. Feynman’s contributions to the advancement of human knowledge are innumerable, and his legacy as one of the world’s most-renowned scientists lives on. 
\newpage

% Begin the pure math section
\addcontentsline{toc}{part}{Pure Mathematics}
\nnimagepage{2020_Pure_Math_Section_Title.pdf}

\nnwallpaper{2020_Pure_Math_Page_Border.pdf}
\def\currentTitleWallpaper{2020_Pure_Math_Title_Page_Border.pdf}

\input{articles/aggarwal-missing-dollar.tex}

\newpage

\input{articles/gao-sprague-grundy.tex}

\newpage

\input{articles/vauclain-pigeonhole.tex}

\newpage

\input{articles/kait-euler.tex}

\newpage

% Begin the philosophy of math section
\nnwallpaper{2020_Philosophy_Page_Border.pdf}
\def\currentTitleWallpaper{2020_Philosophy_Title_Page_Border.pdf}

\addcontentsline{toc}{part}{Applied Mathematics}
\nnimagepage{2020_Philosophy_Section_Title.pdf}

\input{articles/gottlieb-philosophy.tex}

\newpage

\input{articles/fairorth-plato.tex}

\newpage

% Begin the applied math section
\nnwallpaper{2020_Applied_Math_Page_Border.pdf}
\def\currentTitleWallpaper{2020_Applied_Math_Title_Page_Border.pdf}

\addcontentsline{toc}{part}{Applied Mathematics}
\nnimagepage{2020_Applied_Math_Section_Title.pdf}

\input{articles/lee-propeller.tex}

\newpage

\input{articles/greer-lagrange.tex}

\newpage

\input{articles/greer-et-al-m3.tex}

\newpage

% Begin the applied science section
\nnwallpaper{2020_Applied_Science_Page_Border.pdf}
\def\currentTitleWallpaper{2020_Applied_Science_Title_Page_Border.pdf}

\addcontentsline{toc}{part}{Applied Science}
\nnimagepage{2020_Applied_Science_Section_Title.pdf}

\input{articles/demarco-alzheimers.tex}

\newpage

\input{articles/greer-neurology.tex}

\newpage

\input{articles/golecki-fibrillogenesis.tex}

\newpage

\nnwallpaper{2020_Intro_Section_Page_Border.pdf}

\begin{figure}[H]
    \centering
    \vspace*{50pt}
    \includegraphics[scale=1.25]{newtons_notebook_fibonacci_spiral.png}
    \vspace*{25pt}
\end{figure}

    This issue of the \textit{Notebook} included submissions from our own students and faculty as well as alumni and friends of the Haverford community outside the walls of Wilson Hall. Also added in this issue were two new sections containing more complex articles with advanced arguments. The \textit{Notebook} team hopes that the increasing diversity of this journal inspires its readers to pursue STEM with unremitting fervor and renewed dedication. 

\begin{quotation}
\textit{We are at the very beginning of time for the human race. It is not unreasonable that we grapple with problems. But there are tens of thousands of years in the future. Our responsibility is to do what we can, learn what we can, improve the solutions, and pass them on.}
	\begin{flushright}
$\sim$Richard P. Feynman
	\end{flushright}
\end{quotation}

\begin{center}
If you are interested in contributing to the 2021-2022 edition of \textit{Newton’s Notebook},\\ 
please contact Mitav Nayak, Editor for Issue VI.
\end{center}

% v COMMENT THIS OUT BEFORE SENDING THE FINAL VERSION TO PRINT, THEY DON'T WANT THE COVERS
%\newpage
%\nnimagepage{2020_Back_Cover.pdf}

\end{document}
