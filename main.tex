\documentclass[12pt]{article}
\usepackage[utf8]{inputenc}
\usepackage[dvipsnames]{xcolor}
\usepackage[leqno]{amsmath}
\usepackage{amssymb}
\usepackage{units}
\usepackage{wallpaper}
\usepackage{newtons-notebook}
\usepackage{booktabs}
\usepackage{float}
\usepackage{xurl}
\usepackage{listings}
\usepackage{tocloft}
\usepackage{setspace}
\usepackage{titlesec}

\titlespacing\section{0pt}{12pt plus 4pt minus 2pt}{0pt}
\titlespacing\subsection{0pt}{12pt plus 4pt minus 2pt}{0pt}
\titlespacing\subsubsection{0pt}{12pt plus 4pt minus 2pt}{0pt}

\renewcommand{\cftpartfont}{\Large\bfseries}
\renewcommand{\cftsecfont}{\normalsize}

\setlength{\parskip}{\baselineskip}
\renewcommand{\baselinestretch}{1.5}

\addtolength{\oddsidemargin}{-.25in}
\addtolength{\evensidemargin}{-.25in}
\addtolength{\textwidth}{0.5in}
\addtolength{\topmargin}{-.25in}
\addtolength{\textheight}{0in}

\begin{document}
\setcounter{tocdepth}{1}

%\nnimagepage{Title_Page_2019.pdf}

\begin{figure}[H]
    \centering \includegraphics[scale=1]{newton.jpg}
    \\
    \textbf{\large{``To explain all nature is too difficult a task for any one man or even for any one age. `Tis much better to do a little with certainty and leave the rest for others that come after you.'' --Isaac~Newton}}
\end{figure}

\nnwallpaper{2020_Intro_Section_Page_Border.pdf}

\begin{spacing}{1.0}
\tableofcontents
\end{spacing}

\newpage
\subsection*{Mission Statement}
\textit{Newton’s Notebook: The Haverford School STEM Journal} is designed to enhance the interests, talents, and achievements of individuals in mathematics and science and to promote the work of those most passionate about these disciplines. The following articles were written by members and friends of the Haverford community and edited by the \textit{Notebook staff}. We hope these articles inspire readers to further discover the universally beautiful realm of STEM exploration. 
\subsection*{Staff}
{\centering{}
    \textbf{Editor-in-Chief:} Alexander Greer '20
    \\
    \textbf{Assistant Editors:} Gary Gao '21, Mitav Nayak '22, Safa Obuz '21, Shibo Zhou '22
    \\
    \textbf{Designer-in-Chief:} Alexander Greer
    \\
    \textbf{Faculty Advisor:} Dr.\ Mark Gottlieb
    \\}
\subsection*{Featured Polymath: Richard Phillips Feynman (1918-1988)}
Richard Phillips Feynman was an American theoretical physicist known for his work on quantum mechanics, quantum electrodynamics, and particle physics. Born in Queens, New York, in 1918, Feynman excelled in math and science in high school. He attended the Massachusetts Institute of Technology and graduated with a bachelor’s degree in physics in 1939. In early 1943, Feynman was recruited by Robert Oppenheimer to work as a researcher on the Manhattan Project developing the atomic bomb. Soon after his work, Feynman took up a position as a physics professor at Cornell University. There he developed Feynman diagrams, a system of visually representing mathematical expressions describing the interaction of subatomic particles. In 1951, Feynman took up a position at the California Institute of Technology after taking a year-long sabbatical in Brazil. At Caltech, Feynman explored superfluidity and superconductivity, as well as conceiving the possibility of quantum computers. In the early 1960’s, Feynman produced \textit{The Feynman Lectures on Physics}, a lecture series outlining undergraduate physics concepts still used by university students to this day. In 1965, Feynman and two colleagues were awarded the Nobel Prize in Physics for their work on quantum electrodynamics. Feynman’s contributions to the advancement of human knowledge are innumerable, and his legacy as one of the world’s most-renowned scientists lives on. 
\newpage

% Begin the pure math section
\addcontentsline{toc}{part}{Pure Mathematics}
\nnimagepage{Pure_Math_Title_Page.pdf}

\nnwallpaper{Pure_Math_Border.pdf}
\def\currentTitleWallpaper{Pure_Math_Border_Title.pdf}

\input{articles/primes-squared.tex}

\newpage

\input{articles/law-of-large-numbers.tex}

\newpage

\input{articles/search-for-new-primes.tex}

\newpage

\input{articles/ordered-triples.tex}

\newpage

\input{articles/infinitude-of-primes.tex}

\newpage

% Begin the applied math section
\nnwallpaper{Applied_Math_Border.pdf}
\def\currentTitleWallpaper{Applied_Math_Border_Title.pdf}

\addcontentsline{toc}{part}{Applied Mathematics}
\nnimagepage{Applied_Math_Title_Page.pdf}

\input{articles/math-modeling.tex}

\newpage

\input{articles/statistics-in-the-wrong-hands.tex}

\newpage

\input{articles/cryptography.tex}

\newpage

\input{articles/pid.tex}

\newpage

\input{articles/eat-cafe.tex}

\newpage

\begin{figure}[H]
    \centering
    \vspace*{125pt}
    \includegraphics[scale=1.5]{newtons_notebook_fibonacci_spiral.png}
    \vspace*{100pt}
    \\
    \Large If you are interested in contributing to Newton’s Notebook in 2019--2020, contact Alexander Greer, the Editor-in-Chief for Volume IV.
\end{figure}

\newpage

\end{document}
