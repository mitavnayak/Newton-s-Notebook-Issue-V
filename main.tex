\documentclass[12pt]{article}
\usepackage[utf8]{inputenc}
\usepackage[dvipsnames]{xcolor}
\usepackage[leqno]{amsmath}
\usepackage{amssymb}
%\usepackage{units}
\usepackage{wallpaper}
\usepackage{newtons-notebook}
\usepackage{booktabs}
\usepackage{float}
\usepackage{xurl}
\usepackage{listings}
%\usepackage{tocloft}
\usepackage{amsmath}
\usepackage{amsthm}
\usepackage{amsfonts}
\usepackage{amssymb}

\usepackage{tabu}
%\usepackage{multirow}
\usepackage{array}

\usepackage[english]{babel}
\usepackage{longtable}
%\usepackage[table]{xcolor} 
%\usepackage{indentfirst}

%\usepackage{gensymb}

\usepackage{hyperref}
\usepackage{color}
\usepackage[table,xcdraw]{xcolor}

\usepackage[font=small,skip=0pt]{caption}

\usepackage{graphicx}
\graphicspath{ {./assets/} }

\usepackage{setspace}
\usepackage{titlesec}

\usepackage{wrapfig}

% https://tex.stackexchange.com/questions/19660/how-to-make-the-size-of-pdf-output-wider
\setlength{\paperwidth}{8.75in} % set dimension of \paperwidth to add .125 inch on either sidev
\addtolength{\paperheight}{0.25in} % enlarge \paperheight by .25 inch

\titlespacing\section{0pt}{12pt plus 4pt minus 2pt}{0pt}
\titlespacing\subsection{0pt}{12pt plus 4pt minus 2pt}{0pt}
\titlespacing\subsubsection{0pt}{12pt plus 4pt minus 2pt}{0pt}

%\renewcommand{\cftpartfont}{\Large\bfseries}
%\renewcommand{\cftsecfont}{\normalsize}

% deals with space between paragraphs
\setlength{\parskip}{\baselineskip}
\renewcommand{\baselinestretch}{1.4}

% Margin Formatting https://en.wikibooks.org/wiki/LaTeX/Page_Layout
\addtolength{\hoffset}{0in}
\addtolength{\voffset}{0in}
\addtolength{\oddsidemargin}{-.125in}
\addtolength{\textwidth}{0.5in}
\addtolength{\topmargin}{-.125in}

% Theorem Enumeration https://tex.stackexchange.com/questions/371731/versioning-theorem-numbers-with-amsthm
\newcounter{dfnmain}[section]
\newtheorem{dfninner}{Definition}[dfnmain]
\makeatletter
\renewcommand{\thedfninner}{%
  \arabic{dfnmain}.\@arabic{\numexpr\value{dfninner}-1\relax}%
}
\makeatother
\newenvironment{dfn}
 {\stepcounter{dfnmain}\dfninner}
 {\enddfninner}
\newenvironment{dfn*}
 {\dfninner}
 {\enddfninner}
 
% Example formatting https://tex.stackexchange.com/questions/357810/math-example-formatting
\newenvironment{exm}[1]{%
  \par  % start a new paragraph
  \bigskip  % insert some vertical whitespace
  \noindent % no paragraph indentation
  \textbf{Example #1}}{%
  \par%\bigskip % insert another paragraph break and more vert. whitespace
}

\usepackage{tikz}

\newtheorem{theorem}{Theorem}

\theoremstyle{definition}
\newtheorem{definition}{Definition}

\newtheorem{problem}{Problem}

\newtheorem*{solution}{Solution}

\newtheorem*{remark}{Remark}

\newtheorem{assumption}{Assumption}

\begin{document}
\setcounter{tocdepth}{1}

% v COMMENT THIS OUT BEFORE SENDING THE FINAL VERSION TO PRINT, THEY DON'T WANT THE COVERS
% \nnimagepage{2020_Front_Cover.pdf}

\nnwallpaper{2020_Intro_Section_Page_Border.pdf}

\begin{figure}[H]
    \centering \includegraphics[scale=.9]{newton.jpg}
\end{figure}

\begin{center}
    \textit{To explain all nature is too difficult a task\\
    for any one man or even for any one age.\\
    'Tis much better to do a little with certainty\\
    and leave the rest for others that come after you.}\\
    	$\sim$Isaac Newton
\end{center}

\newpage
\begin{spacing}{1}
\tableofcontents
\end{spacing}

\begin{figure}[H]
    \centering
    \vspace*{75pt}
    \includegraphics[scale=1.25]{newtons_notebook_fibonacci_spiral.png}
\end{figure}

\newpage
\subsection*{Mission Statement}
\textit{Newton’s Notebook: The Haverford School STEM Journal} is designed to enhance the interests, talents, and achievements of individuals in mathematics and science and to promote the work of those most passionate about these disciplines. The following articles were written by members and friends of the Haverford community and edited by the \textit{Notebook staff}. We hope these articles inspire readers to further discover the universally beautiful realm of STEM exploration. 
\subsection*{Staff}
{\centering{}
    \textbf{Editors-in-Chief:} Gary Gao '21, Mitav Nayak '22
    \\
    \textbf{Designer-in-Chief:} Mitav Nayak '22
    \\ 
    \textbf{Assistant Editors:} Brian Williams '21, Adamya Aggarwal '22
    \\
    \textbf{Faculty Advisor:} Dr.\ Mark Gottlieb
    \\}
\subsection*{Featured Polymath: John Forbes Nash Jr. (1928-2015)}
John Forbes Nash Jr. was an American mathematician best known for his work and contributions in game theory, partial differential, cryptography, mathematical economics, and differential geometry. A son of an electrical engineer and schoolteacher, Nash was born in 1928 in Bluefield, West Virginia. In high school, he enrolled in various courses in advanced mathematics at a local community college in addition to his school’s general curriculum. Nash went on the graduate with a bachelor’s degree and master’s degree in mathematics from the Carnegie Institute of Technology—now Carnegie Mellon University—when he was just 19 years old. He then attended Princeton University on a scholarship for graduate school and obtained a PhD with a dissertation in non-cooperative games at 22 years of age in 1950. In his dissertation, he included the definition of what later became to be known as the Nash equilibrium, which won him the John von Neumann Theory Prize in 1978, the Nobel Prize in Economic Sciences in 1994, and the Leroy P. Steel Prize in 1999. During much of his life, Nash suffered from mental illness. He joined the Massachusetts Institute of Technology in 1951 to research partial differential equations and teach mathematics, but he resigned in 1959 as a result of his schizophrenia and was admitted to a number of hospitals during the 1960s. In 1970, he was discharged from the hospital for the last time, and eventually returned to Princeton as a senior research mathematician. Nash’s extraordinary life is portrayed in the Academy Award-winning film \emph{A Beautiful Mind} (2001).
\newpage

% Begin the pure math section
\addcontentsline{toc}{part}{Pure Mathematics}
\nnimagepage{2020_Pure_Math_Section_Title.pdf}

\nnwallpaper{2020_Pure_Math_Page_Border.pdf}
\def\currentTitleWallpaper{2020_Pure_Math_Title_Page_Border.pdf}

\input{articles/aggarwal-missing-dollar.tex}

\newpage

\input{articles/gao-sprague-grundy.tex}

\newpage

\input{articles/vauclain-pigeonhole.tex}

\newpage

\input{articles/kait-euler.tex}

\newpage

% Begin the philosophy of math section
\nnwallpaper{2020_Philosophy_Page_Border.pdf}
\def\currentTitleWallpaper{2020_Philosophy_Title_Page_Border.pdf}

\addcontentsline{toc}{part}{Applied Mathematics}
\nnimagepage{2020_Philosophy_Section_Title.pdf}

\input{articles/gottlieb-philosophy.tex}

\newpage

\input{articles/fairorth-plato.tex}

\newpage

% Begin the applied math section
\nnwallpaper{2020_Applied_Math_Page_Border.pdf}
\def\currentTitleWallpaper{2020_Applied_Math_Title_Page_Border.pdf}

\addcontentsline{toc}{part}{Applied Mathematics}
\nnimagepage{2020_Applied_Math_Section_Title.pdf}

\input{articles/lee-propeller.tex}

\newpage

\input{articles/greer-lagrange.tex}

\newpage

\input{articles/greer-et-al-m3.tex}

\newpage

% Begin the applied science section
\nnwallpaper{2020_Applied_Science_Page_Border.pdf}
\def\currentTitleWallpaper{2020_Applied_Science_Title_Page_Border.pdf}

\addcontentsline{toc}{part}{Applied Science}
\nnimagepage{2020_Applied_Science_Section_Title.pdf}

\input{articles/demarco-alzheimers.tex}

\newpage

\input{articles/greer-neurology.tex}

\newpage

\input{articles/golecki-fibrillogenesis.tex}

\newpage

\nnwallpaper{2020_Intro_Section_Page_Border.pdf}

\begin{figure}[H]
    \centering
    \vspace*{50pt}
    \includegraphics[scale=1.25]{newtons_notebook_fibonacci_spiral.png}
    \vspace*{25pt}
\end{figure}

    This issue of the \textit{Notebook} included submissions from our own students and faculty as well as alumni and friends of the Haverford community outside the walls of Wilson Hall. Also added in this issue were two new sections containing more complex articles with advanced arguments. The \textit{Notebook} team hopes that the increasing diversity of this journal inspires its readers to pursue STEM with unremitting fervor and renewed dedication. 

\begin{quotation}
\textit{We are at the very beginning of time for the human race. It is not unreasonable that we grapple with problems. But there are tens of thousands of years in the future. Our responsibility is to do what we can, learn what we can, improve the solutions, and pass them on.}
	\begin{flushright}
$\sim$Richard P. Feynman
	\end{flushright}
\end{quotation}

\begin{center}
If you are interested in contributing to the 2021-2022 edition of \textit{Newton’s Notebook},\\ 
please contact Mitav Nayak, Editor for Issue VI.
\end{center}

% v COMMENT THIS OUT BEFORE SENDING THE FINAL VERSION TO PRINT, THEY DON'T WANT THE COVERS
%\newpage
%\nnimagepage{2020_Back_Cover.pdf}

\end{document}
